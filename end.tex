There is, also, I think, some probability in the view propounded by Thomas Andrew Knight (\href{http://www.wikipedia.org/}{Wikipedia}), that this variability may be partly connected with excess of food. It seems pretty clear that\\

\begin{enumerate}
\item organic beings must be exposed during several generations to the new conditions of life to cause any appreciable amount of variation;
\item that when the organisation has once begun to vary, it generally continues to vary for many generations.
\end{enumerate}

No case is on record of a variable being ceasing to be variable under cultivation. Our oldest cultivated plants, such as wheat, still often yield new varieties: our oldest domesticated animals are still capable of rapid improvement or modification.\\